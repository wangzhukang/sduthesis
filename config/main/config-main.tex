% 导入宏包
\usepackage{amsmath}
\usepackage{amsthm}
\usepackage{amsfonts}
\usepackage{amssymb}
\usepackage{unicode-math}
\usepackage{xeCJK}
\usepackage{xcolor}
\usepackage{geometry}
\usepackage{float}
\usepackage{fancyhdr}
\usepackage{setspace}
\usepackage{bookmark}
\usepackage{booktabs}
\usepackage{graphicx}
\usepackage{caption}
\usepackage{listings}
\usepackage{algorithm2e}
\usepackage{enumitem}
\usepackage{tocloft}
\usepackage{subfig}
\usepackage[
    backend=biber,
    style=gb7714-2015,
    gbnamefmt=givenahead
]{biblatex}
\usepackage{hyperref}


% -------------------文本选项-------------------
% 让已定义的中文字体失效
\let\songti\relex
\let\heiti\relex
\let\kaiti\relex

% 定义中文字体。不建议使用过多种字体,宋体+黑体+楷体已经足够;可以自定义加粗与倾斜的程度,注意数字有取值范围,详情参考xeCJK说明文档
\newCJKfontfamily\songti[
    Path=./fonts/,
    AutoFakeBold=3,
    AutoFakeSlant=0.25
]{SimSun.ttc}

\newCJKfontfamily\heiti[
    Path=./fonts/,
    AutoFakeBold=3,
    AutoFakeSlant=0.25
]{SimHei.ttf}

\newCJKfontfamily\kaiti[
    Path=./fonts/,
    AutoFakeBold=3,
    AutoFakeSlant=0.25
]{SimKai.ttf}

% 后续局部定义字体,后面必须带{}
\newcommand{\song}[1]{{\songti{#1}}}
\newcommand{\hei}[1]{{\heiti{#1}}}
\newcommand{\kai}[1]{{\kaiti{#1}}}

% 后续局部变形处理,对于每种字体定义了加粗、倾斜和加粗倾斜,后面必须带{}
\newcommand{\bfsong}[1]{{\songti\textbf{#1}}}
\newcommand{\bfhei}[1]{{\heiti\textbf{#1}}}
\newcommand{\bfkai}[1]{{\kaiti\textbf{#1}}}
\newcommand{\itsong}[1]{{\songti\textit{#1}}}
\newcommand{\ithei}[1]{{\heiti\textit{#1}}}
\newcommand{\itkai}[1]{{\kaiti\textit{#1}}}
\newcommand{\bfitsong}[1]{{\songti\textbf{\textit{#1}}}}
\newcommand{\bfithei}[1]{{\heiti\textbf{\textit{#1}}}}
\newcommand{\bfitkai}[1]{{\kaiti\textbf{\textit{#1}}}}

% 后续全局变形处理,对于每种字体定义了加粗、倾斜和加粗倾斜,后面不能带{}
\newcommand{\allbfsong}{\songti\bfseries}
\newcommand{\allbfhei}{\heiti\bfseries}
\newcommand{\allbfkai}{\kaiti\bfseries}
\newcommand{\allitsong}{\songti\itshape}
\newcommand{\allithei}{\heiti\itshape}
\newcommand{\allitkai}{\kaiti\itshape}
\newcommand{\allbfitsong}{\songti\bfseries\itshape}
\newcommand{\allbfithei}{\heiti\bfseries\itshape}
\newcommand{\allbfitkai}{\kaiti\bfseries\itshape}

% 设置正文英文字体为Times New Roman
\setmainfont{Times New Roman}

% 设置数学模式下的字体
\setmathfont{XITS Math}
\setmathfont[range=\mathop]{Latin Modern Math}

% 全局默认使用小四号宋体
\AtBeginDocument{\zihao{-4}\songti}

% 全局默认使用1.5倍行距
\setstretch{1.5}
% -------------------文本选项-------------------


% -------------------页面选项-------------------
% 页边距设置
\geometry{
    a4paper,
    left=3cm,
    right=3cm,
    top=2.5cm,
    bottom=2.5cm,
}

% 页面顶部对齐,允许底部有空白
\raggedbottom
% -------------------页面选项-------------------


% -------------------章节标题选项-------------------
% 最多显示三级标题
\setcounter{secnumdepth}{3}

% 章节标题格式与前后距离
\ctexset{
    chapter={
      format=\centering\zihao{3}\allbfhei,
      name={第,章},
      beforeskip=21.6pt,
      afterskip=18pt,
      fixskip=true,
     },
    section={
      format= \zihao{4}\allbfhei,
      name={},
      beforeskip=18pt,
      afterskip=18pt,
      fixskip=true,
     },
    subsection={
            format=\zihao{-4}\allbfhei,
            name={},
            beforeskip=18pt,
            afterskip=18pt,
            fixskip=true,
        }
}
% -------------------章节标题选项-------------------


% -------------------交叉引用选项-------------------
% 自定义颜色
\definecolor{linkdarkblue}{rgb}{0,0.08,0.45}

% 交叉引用颜色选项
\hypersetup{
    colorlinks=true,
    linkcolor=blue,
    urlcolor=linkdarkblue,
    citecolor=red,
}

% PDF属性选项,可有可无
\hypersetup{
    bookmarksnumbered=true,
    pdfstartview=FitH,
}

% 自定义引用命令
\newcommand{\equref}[1]{式\ \eqref{#1}\ }
\newcommand{\tabref}[1]{表\ \ref{#1}\ }
\newcommand{\figref}[1]{图\ \ref{#1}\ }
\newcommand{\subfigref}[1]{{图\ \subref*{#1}\ }}
% -------------------交叉引用选项-------------------


% -------------------图表标题选项-------------------
% 图片存放路径
\graphicspath{{images/}}

% 自定义标题样式
\DeclareCaptionFont{mybfsong}{\allbfsong}
\DeclareCaptionFont{mywuhao}{\zihao{5}}
\DeclareSubrefFormat{mysubref}{#1(#2)}

% 可以自行调整各种距离
\captionsetup[figure]{
    position=below,
    font={mywuhao,stretch=1.0},
    labelfont=mybfsong,
    textfont=mybfsong,
    labelsep=quad,
    margin=40pt,
    belowskip=-4pt,
}
\captionsetup[table]{
    position=above,
    font={mywuhao,stretch=1.0},
    labelfont=mybfsong,
    textfont=mybfsong,
    labelsep=quad,
    margin=40pt,
    belowskip=4pt,
}
\captionsetup[subfloat]{
    position=below,
    font={mywuhao,stretch=1.0},
    labelfont=mybfsong,
    textfont=mybfsong,
    subrefformat=mysubref,
}
% -------------------图表标题选项-------------------


% -------------------数学选项-----------------------
\newcommand{\mye}{\ensuremath{\mathrm{e}}}
\newcommand{\myi}{\ensuremath{\mathrm{i}}}
\newcommand{\myj}{\ensuremath{\mathrm{j}}}
% -------------------数学选项-----------------------
